\section{Nektar-Diffusion}
Proxyapp capable of simulating 2D diffusion with arbitrary symmetric diffusivity tensor, based on Nektar++.  The
equation solved, for diffusivity tensor $D$, temperature T (in the case where the diffusing quantity is heat), and 
time $t$, is

\begin{equation}
\frac{\partial T}{\partial t} = \nabla \cdot \left ( D  \nabla T \right ). 
\end{equation}  

The problem is motivated by the fact that a characteristic of magnetically-confined fusion devices is that diffusive 
transport is strongly anisotropic, with heat transport in the direction of the magnetic field lines being favoured 
(ratios in the components of the diffusivity tensor of up to approximately $10^6$ are physically-motivated).

A typical scenario for the proxyapp is a domain containing linear, parallel magnetic fieldlines making 
angle $\theta$ with the horizontal axis, with a heat source implemented as a Dirichlet boundary condition on the 
left-hand-side vertical boundary, resulting in the diffusive transfer of heat to the lower domain boundary.  The 
interest is in small incident angles as these represent configurations in which the power incident on the wall of a 
nuclear fusion reactor (and hence the likelihood of damage) is minimized.

A number of examples are included, including time-dependent and steady-state problems in the type of geometry
 described above.  There is also an example of 2D semi-annulus in which the field-lines are circular arcs.  See the 
internal report \cite{y3re222} for an presentation of the relevant outputs.

Nektar-Diffusion is publicly available at \url{https://github.com/ExCALIBUR-NEPTUNE/nektar-diffusion}.
