%https://excalibur-neptune.readthedocs.io/en/latest

\chapter{Program Name}\label{sec:PN}
\input{PN/PN}
% Sommerville  - Preface from Fig. 4.17 (p.128) The structure of a requirements document
% Hewitt  - Program Name
\input{approboxw}

\chapter{Business Design}\label{sec:BD}
\input{BD/BD}
% sections 1 to 7 of Hewitt  - Business Design
% Sommerville  - Introduction

\chapter{Requirements Baseline}\label{sec:RB}
The Requirements Baseline (RB) expresses the user requirements for the software.
%The Interface requirements document (IRD) expresses the customer's interface
%requirements for the software to be produced by the developer.
%This document is part of the requirements baseline.
\input{RB/RB}
% Hewitt  - Use Cases
% Sommerville  - User requirements definition
% Smith  - Problem Statement
% Smith  - Requirements  - Introduction
% Smith  - Requirements  - Requirements  - Functional Requirements
% Smith  - Requirements  - General System Description  - User Characteristics

\chapter{Technical Specification}\label{sec:TS}
The Technical Specification (TS)  contains the developer's response to the
requirements baseline. %, and is the primary input to the PDR review process.
%The Interface Control Document (ICD) is the developer's response to the IRD, and is part of the TS.
\input{TS/TS}
% Hewitt  - Application Design  - Standards and policies
% Hewitt  - Application Design  - Guidelines and conventions
% Smith  - Requirements  - Specific System Description  - Problem Description  - Goal Statements
% ICD Interface Control Document
% Smith  - Requirements  - General System Description  - System Constraints

\chapter{Design Justification File}\label{sec:DJF}
The Design Justification File (DJF) is generated and reviewed at all stages of the development
and review processes.  It contains the documents that describe the trade-offs, design choice
justifications, verification plan, validation plan, validation testing specification,
test procedures, test results, evaluations and any other documentation called for to justify
the design of the software.
\input{DJF/DJF}
% Hewitt - ideas - put them here
% Smith  - V+V Plan+Report
% Hewitt  - Application Design  - Testability
% Hewitt  - Application Design  - Monitorability and metrics

\chapter{Design Definition File}\label{sec:DDF}
The Design Definition File (DDF) is a developer-generated file that
documents the result of the design engineering processes.
%The DDF is the primary input to the CDR review process and it contains
%all the documents called for by the design engineering requirements.
\input{DDF/DDF}
% Sommerville  - System architecture
% Sommerville  - System requirements specification
% Sommerville  - System models
% Sommerville  - Appendices
% Smith  - Requirements  - Specific System Description  - Problem Description  - Physical System Description
% Smith  - Requirements  - Specific System Description  - Solution Characteristics Specification
% Hewitt  - Application Design  - Design patterns
% Smith  - Design Specification
% Hewitt  - Application Design  - Scalability and performance
% Hewitt  - Application Design  - Extensibility
% Smith  - Requirements  - Likely Changes
% all of Hewitt  - Data Design
% Smith  - Requirements  - Requirements  - Nonfunctional Requirements
% sections 1 and 2 of Hewitt  - Infrastructure Design

\chapter{Management File}\label{sec:MGT}
The Management File (MGT) is a developer generated file
that describes the management features of the software project notably,
organizational breakdown and responsibilities, work activities breakdown,
selected life cycle, deliveries, milestones and new risks.
\input{MGT/MGT}
% Smith  - Development Plan
% Hewitt  - Infrastructure Design  - Software distribution
% Hewitt  - Business Design  - Governance

\chapter{Maintenance File}\label{sec:MF}
The Maintenance File (MF) is a developer generated file
that describes the planning and status of the maintenance, migration and retirement activities.
\input{MF/MF}
% Sommerville  - System evolution
% Hewitt  - Application Design  - Availability
% Hewitt  - Infrastructure Design  - Maintenance
% Hewitt  - Application Design  - Maintainability

\chapter{Operational Documentation}\label{sec:OP}
The Operational Documentation (OP) consists of the Developer Manual~\Sec{DevMan} and
the User Manual~\Sec{UMan}. It is important that
the user's experience of the software feeds back into the instructions
as to how to use the software, and mechanisms for achieving this end appear
in \Sec{feedback}.
%Everyone hates writing documentation, it gets written badly, and therefore no-one uses it. 
%Therefore no-one funds doing it, and documentation is bad. %Smith
%As with other overheads in software development, expects that work required will reduce over
%time as it becomes standard practice but need to convince people to adopt it in the first place!

\section{Documentation Generally}\label{sec:OP_MGT}
\input{OP/mgt}
\section{Developer Manual}\label{sec:DevMan}
\input{OP/testing}
%\input{OP/DevMan}
%\input{../t33/gamestruct.tex}
\subsection{Object Identification}\label{sec:objdisc}
Douglass~\cite[\S\,5]{douglass} has a description of object analysis which is well-suited
project \nep. His approach is to take the use cases, which for this purpose should
include the \papp s separately, and treat them carefully one after the other using
the strategies indicated in \Tab{objdisc}.
Each \papp\ should be carefully analysed and classes produced from the list of objects
before proceeding to the next.
\input{OP/objdisc}
\section{User Manual}\label{sec:UMan}
ICD Interface Control Document (User Manual)

With video tutorial, quick start, installation, examples. %Smith
Examples use Jupyter - has Python and Julia interfaces. Binder shares a Jupyter notebook so
recipient can immediately execute in a browser

\clearpage
\section{Feedback and Communication}\label{sec:feedback}
\begin{itemize}
\item Matrix chat - Slack?
\item Discourse group
\item Mailing list
\item Suggestion box
\item Weekly community meetings
\end{itemize}

%\input{../../t51/dimred.tex}

% \section{Developer Manual}\label{sec:DevMan}
% \input{DevMan}
% \input{../t33/gamestruct.tex}
% \section{User Manual}\label{sec:UMan}
% ICD Interface Control Document (User Manual)

With video tutorial, quick start, installation, examples. %Smith
Examples use Jupyter - has Python and Julia interfaces. Binder shares a Jupyter notebook so
recipient can immediately execute in a browser

\clearpage
\section{Feedback and Communication}\label{sec:feedback}
\begin{itemize}
\item Matrix chat - Slack?
\item Discourse group
\item Mailing list
\item Suggestion box
\item Weekly community meetings
\end{itemize}

% \input{../t51/dimred.tex}

\chapter{Reference Material}\label{sec:REF}
\input{REF/REF}
% % Sommerville  - Glossary
% % Smith  - Requirements  - Reference Material
% % Smith  - Requirements  - Specific System Description  - Problem Description  - Terminology and Definitions
% % Hewitt  - Infrastructure Design  - Standards and guidelines and conventions
% % Hewitt  -  Acronyms and Symbols

\chapter{Index}\label{sec:IND}
\input{IND/IND}
  % Sommerville
