The success of the project depends on an ability to code up vast numbers
of complicated mathematical expressions containing a wide range of
mathematical variables or symbols, illustrated by over $250$ entries
in Annex B to \cite{pappeqs3}. Coding must be done as automatically, reliably
and unambiguously as possible, starting with source expressions that are
presumed to be in the \LaTeX \ markup language. Unfortunately, the
character set for names of C++ variables is restricted to letters of
upper and lower case plus underscore '\_`. Underscore will be reserved
to separate words, ie. to enable use of 'pothole' convention. There is
thus a challenge as to how to represent the names of variables as set
out using \LaTeX \ conventions, which involve heavy use of the escape
character backslash. To enable conversion, to produce C++ equivalents,
reserve `o' as the escape character, demanding that no mathematical
variable is allowed to use the letter, or its Greek equivalent omicron,
even as a suffix or superfix.

A list in alphabetical order of the conventions as two character codes
is provided by \Tab{twoclatex}.

\subsubsection{Two character variables}\label{sec:two-character-variables}

Each keyboard character will be represented by one or two lowercase
letters, normally those which form the first two letters of its name,
ommitting `o', thus :

\begin{itemize}
\item `a' will represent `a'
\item `aa' will represent `A'
\item `as' will represent '*' (for asterisk)
\item `bl' will represent bracket on left {[}
\item `br' will represent bracket on right {]}
\item `pl' will represent parenthesis on left \{
\item `pr' will represent parenthesis on right \}
\item `ti' will represent tilde
\item `ci' will represent circumflex
\item `sq' will represent single quote
\item `dq' will represent double quote
\item `st' will represent stop or dot
\item `ds' will represent double stop or double dot
\item `pu' will represent `+'
\item `mn' will represent `-'
\item `gt' will represent $>$
\item `lt' will represent $<$
\item `vb' will represent $|$
\end{itemize}

Should it be necessary to use `o', then `ooo' will represent lowercase
`o' and `oooo' will represent `O'.

Greek lowercase letters and other special characters will similarly be
represented by two lowercase letters, apart from `o', thus:

\begin{itemize}
\item `al' will represent $\alpha$
\item `be' will represent $\beta$
\item `me' will represent $\omega$
\item `ar' will represent arrow to right $\rightarrow$
\item `pa' will represent parallel $\|$
\item `pe' will represent perpendicular $\perp$
\item `un' will represent underscore \_
\end{itemize}

Variant Greek letters will begin with `v', followed by the first letter
of their transliteration into Roman letters, and uppercase Greek letters
will be represented by the first and third letters, except for $\Pi$,
$\Phi$ and $\Psi$ where this is not possible, thus:

\begin{itemize}
\item `ve' will represent $\varepsilon$
\item `dl' will represent $\Delta$
\item `mg' will represent $\Omega$
\item `py' will represent $\Pi$
\item `pf' will represent $\Phi$
\item `pj' will represent $\Psi$
\end{itemize}

If storage of an expression is required, the following will be useful,
thus:

\begin{itemize}
\item `pt' denotes $\partial$
\item `ml' denotes multiplication
\item `dv' denotes division
\end{itemize}

A single letter followed by a digit will have that as a suffix, thus:

\begin{itemize}
\item `n1' will denote $n_1$
\end{itemize}

\subsubsection{Escape sequences}\label{sec:escape-sequences}

Use of different fonts will be denoted by `o' followed by one or two
digits, preceding the above codes, thus :

\begin{itemize}
\item `o1' will denote uppercase, for use in conjunction with Greek alphabet
\item `o2' will denote bold(math)
\item `o3' will denote calligraphic, (math)cal
\item `o4' will denote sans-serif, (math)sf
\item `o5' will denote typewriter, (math)tt
\end{itemize}

Some of the higher integer values might be used to denote members of the
same `namespace', cf.~the way sf is used to denote neutral quantities.

Positioning of suffixes and prefixes will be denoted by `o' followed by
a single letter, thus:

\begin{itemize}
\item `os' indicates underneath, S for South
\item `on' denotes above, N for North
\item `oe' denotes suffix, E for East
\item `ow' denotes prefix, W for West
\item `or' denotes superfix, R for Right
\item `ol' denotes preceding superfix, L for Left
\end{itemize}

Other \LaTeX \ commands will simply see their leading backslash replaced by
`o', thus:

\begin{itemize}
\item `onabla' indicates $\nabla$
\item `otimes' indicates $\times$
\end{itemize}
