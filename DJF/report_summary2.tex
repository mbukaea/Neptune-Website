
For Deliverable 4, UKAEA produced three milestone reports, namely 
\begin{itemize}
\item CD/EXCALIBUR-FMS/0066-M4.1   Support High-dimensional Procurement
\item CD/EXCALIBUR-FMS/0061-M4.2   2-D Model of Neutral Gas and Impurities 
\item CD/EXCALIBUR-FMS/0062-M4.3   High-dimensional Models Complementary Actions 2 
\end{itemize}
The first (Report~66)  supports the usefulness of the call for development
of higher dimensional elements, suitable for solution of a continuum
kinetic model of plasma, by demonstrating Nektar++ solution of a~$1d1v$
model. The second (Report~61)  considers the use of the Julia language as
a means to coding  particle models of plasma and neutral gas at HPC,
with broadly favourable conclusions. The third and final (Report~62) 
outlines critical physics expected to require treatment using
particle-based and/or Monte-Carlo methods.  Its principal content
examines how best to treat inter-processor communication of
particle-based information at the Exascale.
There is also a brief description of a \papp \ for the exploration
of $1d1v$~solution by particles.

For Deliverable 5, UKAEA produced two milestone reports, namely 
\begin{itemize}
\item CD/EXCALIBUR-FMS/0040-M5.1   Management of external research. Supports UQ Procurement 
\item CD/EXCALIBUR-FMS/0063-M5.2   Selection of techniques for Uncertainty Quantification 
\end{itemize}
The first (Report~40)  begins with a reminder of the high
level of ``noise" in tokamak data, and the kind of comparisons
with simulation required.  It is pointed out that although
spline interpolants are provably optimal, they may perform
poorly for classes of functions relevant to the tokamak edge.
The main content is the use of the VECMA toolkit for UQ of
BOUT++ and Nektar++ for two 2-D fluid dynamical models.
There is also derivation of simplified model by dimension
reduction, by integration in one coordinate and/or
by use of the Lie derivative.
The report finishes with a summary of
UQ-related PhD projects sponsored by \nep.

The second (Report~63)  pursues
the use of splines for \nep, indicating utility in the case of noisy data,
and explores the use of Gaussian processes, including derivation
of key formulae, and highlighting their strengths relative to splines.
There is a brief comparison with neural network surrogates in an annex.

For Deliverable 6, UKAEA produced two milestone reports, namely 
\begin{itemize}
\item CD/EXCALIBUR-FMS/0051-M6.1   Finite Element Models: Complementary Activities I 
\item CD/EXCALIBUR-FMS/0064-M6.2   Finite Element Models Complementary Actions 2 
\end{itemize}
The first (Report~51)  details the application of the Nektar++ spectral / hp finite-element software to the classic problem 
of two-dimensional vertical natural convection - physically a model for the heat transfer that takes place within the 
cavity of a double-glazing unit and also relevant to heat transport in a plasma.  A brief survey of results from the 
literature is followed by a numerical investigation showing transitions between conducting, laminar convective, and 
turbulent regimes.  Small extensions to the existing Nektar++ framework, aimed at extracting engineering-relevant 
quantities such as the maximum near-wall temperature, are given.  The second (Report~64)  builds on these results with a 
quantitative comparison to the well-established MIT numerical convection benchmark and also the reproduction of some 
detailed flow-fields from a recent publication; excellent agreement between results from Nektar++ and those from the 
literature is obtained in both cases.  Also included in this report is a preliminary study of a numerical 
implementation of discrete exterior calculus with a demonstration of spectral convergence for a simple test problem, 
work motivated by the favourable properties of such schemes when coupled to particle kinetic codes.

For Deliverable 7, UKAEA produced three milestone reports, namely 
\begin{itemize}
\item CD/EXCALIBUR-FMS/0052-M7.1   Literature review for Call T/AW086/21: Mathematical Support for Software Implementation 
\item CD/EXCALIBUR-FMS/0053-M7.2   Code coupling and benchmarking 
\item CD/EXCALIBUR-FMS/0065-M7.3   Software Support Complementary Actions 2 
\end{itemize}
The first of these (Report~52)  is a literature review performed to support the Call T/AW086/21: Mathematical Support for 
Software Implementation. It describes recent advances in algorithm development for hyperbolic and elliptic equations.  
In particular, the report describes developments in IMEX schemes, Deferred Correction methods, Asymptotic Preserving 
(AP) methods, and Variable Stepsize, Variable Order (VSVO) timestepping for hyperbolic problems, and AP methods and 
nested solvers for elliptic problems.

The second (Report~53)  provides a commentary on the present state of Exascale hardware and software, and discusses the 
available tools and technologies available for benchmarking and code coupling. The hardware and software landscape of 
HPC systems is becoming increasing diverse, with a proliferation of vendors and different technologies. To perform well 
at Exascale, software will likely need be able to target multiple heterogeneous systems. In such an environment, it is 
crucial for developers to have access to benchmarking infrastructure to measure performance and highlight regressions. 
This environment  and the separation of concerns approach taken to navigate it  necessitates the development of 
discrete \papp s which will need to be drawn together into a single software suite. Thus the issue of code coupling 
will also be important at Exascale.

The final (Report~65)  describes aspects of coordination within the \nep \   project not covered in previous reports, namely, 
the development of a GitHub repository for infrastructure code and project planning; the development of a project 
website for knowledge transfer within \nep \  ; and a description of collaborations arising from \nep \  -related 
interactions, including the Fusion Modelling Use Case working group established create a connection between Project 
\nep \   and the wider \exc \   programme.
